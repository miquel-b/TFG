% vim: set spell:
\documentclass[]{article} 
\usepackage[english]{babel} %,catalan
\usepackage{csquotes}
\usepackage{graphicx} % Imatges
\usepackage{fancyhdr}
\usepackage{ragged2e}
\usepackage{hyperref}
\usepackage{textcomp}
\usepackage[x11names, rgb]{xcolor}
\usepackage{listings}
\usepackage{enumitem}
\usepackage{pdfpages}
\usepackage[backend=biber,citestyle=chem-acs]{biblatex}
%\usepackage{biblatex}
\addbibresource{TFG.bib}
%\bibliography{TFG}
%\bibliography{MemoriaPractiques}
\usepackage{setspace}
\usepackage{rotating}
\usepackage{censor}
\usepackage[paper=portrait,pagesize]{typearea}
\usepackage{pdflscape}
\usepackage{geometry}
    \geometry{
    a4paper,
    inner=3.75cm,
    outer=1.75cm,
    top=3cm,
    bottom=3.5cm,
    headheight=1.5cm,
    footskip=42pt}

\hypersetup{
    colorlinks,
    linkcolor={blue!50!black},
    citecolor={blue!50!black},
    urlcolor={blue!80!black}
}
   

\title{KAIZEN STRATEGIES FOR IMPROVING PLANNING AND PRODUCTION FLOW IN PHARMACEUTICAL ENVIRONMENTS}
\author{Miquel Borrell I Candelich}
\date{August 2023}


\setlength{\parindent}{0pt}

\lstset{
    % How/what to match
    sensitive=true,
    % Border (above and below)
    frame=lines,
    % Extra margin on line (align with paragraph)
    xleftmargin=\parindent,
    % Put extra space under caption
    belowcaptionskip=1\baselineskip,
    % Colors
    backgroundcolor=\color{sbase3},
    basicstyle=\color{sbase00}\ttfamily\scriptsize,
    keywordstyle=\color{scyan},
    commentstyle=\color{sbase1},
    stringstyle=\color{sblue},
    numberstyle=\color{sred},
    identifierstyle=\color{sorange},
    % Break long lines into multiple lines?
    breaklines=true,
    % Show a character for spaces?
    showstringspaces=false,
    tabsize=2
}

\usepackage[T1]{fontenc}
\usepackage[normalem]{ulem}

\newif\ifconfid
\confidtrue                   % comment this line out to unveil all hidden text
\colorlet{confidcolor}{black} % change the colour of boxes here (e.g. "red")


% ----------- implementation details -----------
\makeatletter

% from Martin's Scharrer's answer (https://tex.stackexchange.com/a/16004/21891)
\def\confid@UL@putbox{%
  \ifx\UL@start\@empty%
  \else % not inner
    \vrule\@width\z@ \LA@penalty\@M
    {\UL@skip\wd\UL@box \UL@leaders \kern-\UL@skip}%
    \phantom{\box\UL@box}%
  \fi
}

\newcommand\confidentiel{}%

\ifconfid
\renewcommand\confidentiel[1][confidcolor]{%
 \bgroup%
 \let\UL@putbox\confid@UL@putbox%
 \markoverwith{\hbox to.01em{\hss\textcolor{#1}{|}\hss}}\ULon%
}
\fi

\lst@AddToHook{Init}{%
  \ifconfid%
    \lstset{moredelim=[is][\confidentiel]``,keepspaces}%
  \else
    \lstset{moredelim=**[is][]``}%
  \fi
}
\makeatother
% ----------- end of implementation details -----------

\begin{document}
\onehalfspacing
%Peus de pàgina
\pagestyle{fancy}
\fancyfoot[R]{\includegraphics[]{footer4.png}}
\fancyfoot[C]{\thepage}
\fancyhead[R]{Kaizen Strategies For Improving Planning And Production Flow In Pharmaceutical Environments}
\fancyhead[L]{}



\begin{titlepage}

\includegraphics[width=0.4\textwidth]{LogoUPC.png}
\begin{center}
       \vspace*{1cm}

       \textbf{Informe de pràctiques}

       \vspace{0.5cm}
        \textbf{Doble titulació en Enginyeria Química i Enginyeria Electrònica Industrial i Automàtica}\\
            
       \vspace{1.5cm}

       \textbf{Entitat col·laboradora: GasN2}\\
       Ubicació: Sentmenat\\

\end{center}
\includegraphics[]{BannerEEBE.png}
       
       \vfill
\textbf{Autor}:Miquel Borrell i Candelich\\
\textbf{Tutor EEBE:}Roque Lopez Paricio\\
\textbf{Tutor entitat col·laboradora:} Jordi Pujol\\
\textbf{Període de pràctiques:} Des de 3/07/2023 fins a 15/09/2023\\
            
       \vspace{0.8cm}
            
\end{titlepage}


\newpage
\newpage

\justifying
\section{Abstract}\label{abstract}
\newpage

\section{Resum}\label{resum}
\newpage

\section{Resumen}\label{resumen}
\newpage

\section{Glossary}
\newpage

% TABLE OF CONTENTS
\tableofcontents
\clearpage
\newpage

\section{Preface}\label{preface}
\subsection{The KAIZEN Philosofy}\label{philosofy}
After World War II, Toyoda Kiichiro, then president of the Toyota Motor Company, saw the need
to catch up with America to assure the survival of the automobile industry of Japan. Taiichi Ohno,
Toyota’s engineer, had a challenge: how to build a diversified line of products with the limited
equipment Toyota possessed at the time. Instead of using batch production, he designed a system
of integrated production that used one-piece flow, demonstrating that a trade-off between quality
and productivity does not need to exist (Netland and Powell, 2016). Aligned with other practices,
Toyota Production System (TPS) was born to defy the premise that bigger batches mean lower
costs.

TPS, which ultimate goal is the absolute elimination of waste, is based in two pillars: just-in-
time (JIT) and autonomation (automation with a human touch). The lack of the necessary capital
resources to support high inventory levels was solved with the implementation of a just-in-time
pull system. JIT means "to produce the necessary units at the necessary quantities at the necessary
time". A company establishing this flow throughout can approach zero inventory. With autonoma-
tion, Toyota intended to create a built-in automatic checking system against small abnormalities.
With this prevention mechanism, defective products were not produced. That was not the only
advantage. While the machine is working properly, employees are liberated. Consequently, one
worker could attend multiple machines, increasing production efficiency. Furthermore, autonoma-
tion came as a facilitator of improvement. Stopping the machine creates awareness and, through
the total comprehension of the problem, improvement measures can be taken. (Ohno, 1988)
Lean, based on TPS, is a methodology that accentuates customer needs, to improve quality
and reduce costs through continuous improvement and employee involvement (Graban, 2008).
Kaizen, a Japanese word, refers to the concept of continuous improvement. It does not intend
to be mistaken as innovation or disruption and elevated costs. On the other hand, Kaizen is about
small and subtle improvements. It is a low-cost low-risk process that assures incremental progress
and sustainable changes in the long term.

At the beginning of the 21st century, Toyota Motor Company surpassed General Motors and
became the world leader in automobile production, expanding and elevating the Kaizen concept
for its key role in the company’s success. Kaizen’s methodology is based on five
fundamental principles described in \citeauthor{massaki_gemba_2015}'s book \cite{massaki_gemba_2015} :
\begin{itemize}
  \item Create customer value
    \subitem Value is what the customer is willing to pay for. Using a market-in approach to make
informed decisions based on what the customers want and offer them that in the best way
possible, improving customer experience.
  
  \item Create flow efficiency
    \subitem Flow efficiency can be obtained through the elimination of three Ms: Muda (waste), Mura (unevenness) and Muri(overburden).
    The concept of Muda primarily originated from Taiichi Ohno’s production philosophy in the
    early 1950s. He defined, in the industrial context he was inserted, seven sources of Muda:
    overproduction, waiting, inventory, motion, transportation, over-processing and defects. All
    the activities that do not add value to the process are considered Muda. Fujio Cho, former
    President of Toyota, defined waste as "anything other than the minimum amount of equipment, materials, parts, space and worker’s time, which are absolutely essential to add value
    to the product". Eliminating waste is the most effective way to increase productivity and
    decrease costs.
    Although Muda has gained bigger awareness, comprehending all three concepts allows for
    a greater understanding of Lean (Netland and Powell, 2016). Mura is the variation in a
    process that is not caused by the final customer, for example highs and lows in the planned
    production due to the production system or the irregular work rhythm. Leveling production
    using JIT can eliminate this irregularities, avoiding work spikes and long waiting times for
    workers. Muri means to overburden equipment or workers, demanding a faster rhythm for
    a longer period than expected. While in equipment it can provoke defects and failures, in
    people it can result in safety issues.

  \item Be Gemba oriented
    \subitem Gemba is the place where value is added. Problems should be identified and solved there, at their root cause.
    Central to Kaizen is \textit{genchi genbutsu} (go and see for yourself). Going to the Gemba means going to the place where the action really happens, to be able to thoroughly observe the reality of the processes.
  \item Empower people
    \subitem Respect for people is a fundamental principle of the Kaizen philosophy. Kaizen recognizes that people are not the problem, processes are. It intends to improve every day, everywhere, with everyone, training all employees and encouraging people to think and share their own improvement ideas, systematically. It is in management’s biggest interest to create the best possible conditions for everyone to do the best possible work.
  \item Be scientific and transparent
    \subitem As important as identifying problems, is knowing how to explain and substantiate one’s findings, supporting them with data. Visual Management is a basal tool of Kaizen, that highlights problems and allows for a quicker reaction to them.
\end{itemize}
\clearpage
\newpage

\section{The Problem at hand}
\subsection{Introducing the company}

Now that we have introduce the base methodology of Kaizen we must understand the focus of this work.
The project describe has been implemented in a Pharmacology Company located on the Catalonia area. It focuses primarily on providing Contract Manufacturing and Packaging Services for Medicines and Dietary Supplements, both for Human and Veterinary use.

The facilities are equipped with state-of-the-art machinery and have a total area of 7.200 m², distributed across two facilities, one dedicated to the manufacturing of pharmaceuticals and the other to nutraceuticals.

In total the company has over 400 employees and about 80\% of the workforce is established on the shop floor.

There are several manufacturing areas and work lines:

\begin{itemize}
  \item TMP (Raw materials handaling)

\end{itemize}
\clearpage
\newpage
\printbibliography[heading=bibnumbered]
\end{document}
