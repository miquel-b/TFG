\usepackage[acronym]{glossaries}
\makeglossaries

\newglossaryentry{andon}{
    name={Andon},
    description={Visual management tool, originating from the Japanese word for ‘lamp’. Most commonly, andons are lights placed on machines or on production lines to indicate operation status, notifying management, maintenance and other workers. Andons are commonly colour-coded green (normal operations), yellow (changeover or planned maintenance), and red (quality or process issue, machine down) often combined with an audible signal such as music or alarms. The andon concept can also be used to show project status with the colors green, yellow, or red meaning on track, slipping, late or to indicate general business performance as in on target, behind target, target missed.}
}

\newglossaryentry{controlchart}{
    name={Control Chart},
    description={A chart with upper and lower control limits within which a machine or process is “in control”. Frequently a centerline, midway between the two limits, helps detect trends toward one or the other. Plotting critical measurements on the chart shows when a machine or process has gone “out of control” and must be adjusted. It is one of the Basic Seven Tools of Quality.}
}

\newglossaryentry{externalcustomer}{
    name={External Customer},
    description={An end-user who purchases a company’s products or services but is not an employee or part of the organization. The goal of world-class companies is to “continually delight” this customer, thus creating “an increasing affection” for its products and services.}
}

\newglossaryentry{pdca}{
    name={PDCA (Deming Cycle)},
    description={The Deming Cycle, or PDCA Cycle (also known as PDSA Cycle), is a continuous quality improvement model consisting out of a logical sequence of four repetitive steps for Continuous Improvement and learning: Plan, Do, Study (Check) and Act.}
}

\newglossaryentry{gwqc}{
    name={Group-wide Quality Control (GWQC)},
    description={A system of continuing interaction amongst all elements, including suppliers, responsible for achieving the continuously improving quality of products and services that satisfy customer demand.}
}

\newglossaryentry{kaizen}{
    name={KAIZEN™},
    description={A Japanese term meaning change for the better. KAIZEN™ is a gradual and long-term approach to achieve small, incremental changes in processes in order to improve efficiency and quality. KAIZEN™ was popularized by Masaaki Imai in his book ‘KAIZEN™’: The Key To Japan’s Competitive Success.}
}

\newglossaryentry{kamishibai}{
    name={Kamishibai},
    description={Kamishibai is a way of telling stories that originated in Japan. Kamishibai cards are used, in a business context, as a visual control to perform audits to processes. The result of the verification is visually displayed (green/red).}
}

\newglossaryentry{kobetsukaizen}{
    name={Kobetsu KAIZEN™},
    description={The word Kobetsu originated in Japan and means “focused,” with Kobetsu KAIZEN™ referring to focused KAIZEN™. It is a structured problem-solving methodology used to achieve different goals such as: reduce defects or errors, costs and delivery times; increasing productivity or the security at the workstation.}
}

\newglossaryentry{lean}{
    name={Lean},
    description={In English this term literally means that it contains no fat, this is, lean. When applied to the management context, it means a strategy that aims to reduce or eliminate excess/ waste.}
}

\newglossaryentry{nonstatisticalqc}{
    name={Non-statistical Quality Control},
    description={Most of quality control is non-statistical, particularly that portion which has to do with human resources. Elements are self-discipline, morale, communications, human relations, and standardization. Statistics are only one tool in Quality Control and are of limited use with regard to human beings and methods.}
}

\newglossaryentry{policy}{
    name={Policy},
    description={In Japan, this term is used to describe long – and medium-range – management priorities, as well as annual goals or targets. Policy is composed of both goals and measures (ends and means). Goals (control points) are usually quantitative figures established by top management, such as sales, profit, and market share. Measures (Check Points) are the specific action programs designed to achieve these goals.}
}

\newglossaryentry{qfd}{
    name={Quality Function Deployment},
    description={A system whereby customer requirements, known as true quality characteristics are translated into designing characteristics, known as counterpart characteristics, and then deployed into such sub-systems as components, parts and production processes to develop new products precisely designed to meet customer needs. QFD is one of the seven KAIZEN™ Systems.}
}

\newglossaryentry{sdca}{
    name={SDCA (Standardise, Do, Check, Act)},
    description={A refinement of the PDCA cycle aimed at stabilization of production processes prior to making attempts to improve.}
}

\newglossaryentry{takttime}{
    name={Takt Time},
    description={A vital element in balancing single-piece production flows, Takt Time is calculated by dividing the total daily customer demand in completed units (television sets, automobiles, can openers, and the like), by the total number of production minutes or seconds worked in a twenty-four hour period.}
}

\newglossaryentry{tps}{
    name={Toyota Production System (TPS)},
    description={A methodology that resulted from over 50 years of KAIZEN™ at Toyota. TPS is built on a foundation of Leveling, with the supporting pillars of Just-in-Time and Jidoka.}
}

\newglossaryentry{variabilitycontrol}{
    name={Variability Control},
    description={A KAIZEN™ concept which is often called Ask why five times because it seeks through curious questioning to arrive at the root cause of a problem so that the problem can be eliminated once and for all.}
}

\newglossaryentry{warusakagen}{
    name={Warusa-Kagen},
    description={A term in TQC that refers to things that are not yet problems, but are still not quite right. They are often the starting point of improvement activities because if left untended they may develop into serious problems. In gemba, it is usually the operators who first notice Warusa-Kagen, and who therefore are on the front line of improvement.}
}

\newglossaryentry{upstreammanagement}{
    name={Upstream Management},
    description={A KAIZEN™ concept and process whereby, through continuous improvement, defects are eliminated farther and farther upstream in the production process, first in inspection, then in the line, then in development.}
}

\newglossaryentry{coreprocess}{
    name={Core Process},
    description={Key activity or cluster of activities which must be performed in an exemplary manner to ensure a firm’s continued competitiveness because it adds primary value to an output.}
}

\newglossaryentry{commoncauses}{
    name={Common Causes},
    description={In quality control, inherent source of variation that is 1. random, 2. always present, and 3. affects every outcome of the process. The common cause is usually traced to an element of the system that can be corrected only by the management. Also called assignable cause.}
}

\newglossaryentry{internalcustomer}{
    name={Internal Customer},
    description={Any member of an organization who relies on assistance from another to fulfill the job duties, such as a sales representative who needs assistance from a customer service representative to place an order.}
}

\newglossaryentry{eighty20rule}{
    name={Eighty-twenty Rule},
    description={Refers to the Pareto principle stating that for many events roughly 80% of the effects come from 20% of the causes}
}

\newglossaryentry{foundationkaizen}{
    name={Foundation of KAIZEN™},
    description={The three principles and seven concepts of KAIZEN™ which serve as a foundation for the systems and tools required for the implementation of Continuous Improvement and Total Quality Management, and which shape the culture and thinking of an organization’s leadership.}
}

\newglossaryentry{hanedashi}{
    name={Hanedashi},
    description={Automatic parts ejection. Parts may be manually inserted into a machine, but when the cycle is complete the processed part is automatically ejected so the operator can simply insert the new work and move the ejected part on to the next process, thus reducing his/her cycle time.}
}

\newglossaryentry{kaizenculture}{
    name={KAIZEN™ Culture},
    description={An organisational culture based on the three principles – Process and Results, Systemic Thinking and Non-judgmental/Non-Blaming.}
}

\newglossaryentry{kanban}{
    name={Kanban},
    description={A materials requirement planning tool in the Just-in-Time production and inventory control system developed by Toyota. Kanban is often seen as a central element of Lean manufacturing and is probably the most widely used type of Pull signaling system. Kanban stands for a visual sign (Kan- card, Ban- signal). On the basis of automatic replenishment (through signal cards that indicate when more goods are needed) the flow of goods with outside suppliers and within the factory and the customers, is regulated, this system is called “Kanban”.}
}

\newglossaryentry{kpi}{
    name={KPI},
    description={The English acronym for key indicators, the key indicators of a given process or activity.}
}

\newglossaryentry{milkrun}{
    name={Milkrun},
    description={A method of organising transport in standardised cycles in which a vehicle passes through several collection/ delivery points with a fixed frequency. It functions as a subway line that meets fixed times with certain frequencies.}
}

\newglossaryentry{muda}{
    name={Muda},
    description={Japanese word for Waste and a key concept in the TPS as one of the three types (muda, Mura [Irregularity or Unevenness] and Muri [Strain]) of deviation from optimal allocation of resources.}
}

\newglossaryentry{mizusumashi}{
    name={Mizusumashi (Water-Spider)},
    description={A person who manages all the logistical work of bringing components, raw materials, etc. in small
::contentReference[oaicite:0]{index=0}
 
